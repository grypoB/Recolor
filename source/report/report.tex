\documentclass[a4paper]{report} % set paper size

\usepackage[utf8]{inputenc}
\usepackage {url}
\usepackage[top=2.54cm, bottom=2.54cm, left=2.54cm, right=2.54cm]{geometry} % set margin
\usepackage{amsfonts} % for set names
\usepackage{amsmath} % for equation system
\usepackage{amsthm} % for theorem block
\usepackage{fixltx2e} % for subscript
\usepackage{fancyhdr} % for footer/headline modification
\usepackage{xcolor}
\usepackage[ruled,vlined]{algorithm2e} % for algorithm integration

\pagestyle{fancyplain} % for footing modification on all pages
%\fancyhf{}
\renewcommand{\headrulewidth}{0pt} % remove decorative lign
\fancyhead[L]{Alexandre Devienne BA1 EL 2014}

\begin{document}

\section*{Recolor}
Describe program

show abstraction/re-usage of fct

analyse complexity of filtrage

\begin{algorithm}
    \KwData{$image$ seuilée, $nbF$ number of filtrage}
    \KwResult{Image filtrée}
    \BlankLine
    Let $temp$ a matrix of colors and size the original image\;
    Let $pixel$, $neighboor$ and $val$ colors\;
    Let $T$ a dictionnary as described in function below and of size $3$\;

    \BlankLine
    Set the border of $temp$ to black\;
    \For{$i \leftarrow 1$ \KwTo $nbF$}{
        \For{each $pixel$ in $image$ (border excepted)}{
            $val \leftarrow $ UNASSIGNED\;
            reset $T$ (remove all entry and set all integers to 0)\;
            \For{each $neighboor$ of $pixel$} {
                \If{$val = $ UNASSIGNED} {
                    $val \leftarrow $ \emph{insertInDictionnary}($neighboor$, $T$)\; 
                }
            }
            \If{$val = $ UNASSIGNED} {
                $val \leftarrow$ black \;
            }
            Set corresponding $pixel$ of $temp$ to $val$\; 
        }
        $image \leftarrow temp$\;
    }

\BlankLine
\SetKwProg{Fn}{Function}{}{}
\Fn{insertInDictionnary($color$, $T$)}
{
\KwIn{$color$ to insert in dictionnary $T$ of limited size, and which can contain a integer for each entry (colors)}
\KwOut{Color of resulting cell, or UNASSIGNED if cannot decide yet}
\BlankLine
\uIf{$color$ already in $T$} {
    increase corresponding integer by 1\;
    \If{$integer > 6$} {
        \KwRet $color$
    }
}
\uElseIf{$T$ full} {
    \KwRet black\;
}
\Else
{
    Store $color$ in new entry of $T$\;
    Set correponding integer to 1\;
}
\KwRet UNASSIGNED
}
\caption{Filtrage algorithm}
\end{algorithm}

\end{document}
